\documentclass[11pt]{article}

\title{Flatland}
\author{Ryan Kepler Murphy}
\date{\today}

\begin{document}

% Title page
\maketitle	
\pagebreak

% Table of contents
\tableofcontents
\pagebreak

% Paper content

% Problem background
\section{What is this problem?}
\subsection{What is Flatland?}
The Flatland competition seeks to address the problem of automated train scheduling and rescheduling, a major challenge
for modern railway systems. It does so by providing a simplified two-dimensional grid world environment to allow for fast experimentation of new approaches to this problem \cite{monylascscbhwaegeibavistsasp20a}. 

\subsection{What work is similar to Flatland?}
In essence, the Flatland problem is a vehicle scheduling or vehicle rescheduling problem.  The vehicle scheduling problem (VSP) is [definition].   \\

The vehicle rescheduling problem (VSRP) arises when a previously-scheduled trip is disrupted due to interruptions such as a traffic collision, a medical emergency, or a vehicle breakdown \cite{limibo07a}.  Trips in the Flatland environment may be disrupted by randomly-assigned vehicle breakdowns, each of which stops a train in its current location for an unforeseen duration.  Ideally, scheduled trips that are affected by a breakdown should be rescheduled in such a way that there are minimal impacts to the original plan. \\

\subsection{What is multi-agent pathfinding?}
Multi-agent pathfinding (MAPF) is a planning problem in which agents in a shared environment must find routes to their respective destinations without incurring collisions \cite{silver05a}. \\

\begin{itemize}
  \item Explanation
  \item Differences — four-connected, eight-connected, graphs
  \item Hypergraphs
\end{itemize}

% Potential approaches
\pagebreak
\section{How can this problem be addressed?}
\subsection{Which methods have been used?}
\begin{itemize}
  \item Reinforcement learning
  \item Deep learning
  \item Previous winners
\end{itemize}

\subsection{What is answer set programming?}
\begin{itemize}
  \item Definition
  \item How can it help in this case?
\end{itemize}


% The problem workflow
\pagebreak
\section{What does the problem workflow comprise?}
\subsection{Overview of the fundamental pieces}
\begin{itemize}
  \item Environment
  \item Agents
  \item Breakdowns
  \item Observation types
\end{itemize} 

\subsection{Environment}
Explain fundamentally what they are, but also how we define them.
\begin{itemize}
  \item Track types
  \item Cities and stations
  \item Transitions
\end{itemize} 

\subsection{Agents}
Explain fundamentally what they are, but also how we define them.

\subsection{In this paper}
We start by looking at the simplest of workflows: a single agent in a small environment with only straight tracks and dead ends.

\subsection{Technical details}
\begin{itemize}
  \item How does Flatland represent environments?
  \item How does Flatland generate random environments?
  \item How do we provide a connection between Flatland output and clingo?
\end{itemize} 


\bibliography{references}
\end{document}